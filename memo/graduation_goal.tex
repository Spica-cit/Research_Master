\documentclass[a4paper,12pt]{ltjsarticle}
\usepackage{luatexja}
\usepackage{luatexja-fontspec}
\setmainjfont{IPAexMincho}
\setmainfont{Times New Roman}
\setsansjfont{IPAexMincho}
\setsansfont{Times New Roman}
\usepackage{geometry}
\geometry{margin=15mm}

\usepackage{enumitem}
\usepackage{hyperref}
\usepackage{graphicx}
\usepackage{amsmath, amssymb}

\renewcommand{\baselinestretch}{1.2}
\setlength{\parskip}{0.5em}
\setlength{\parindent}{1em}

\begin{document}

  \begin{center}
    {\huge 卒業までの活動方針}
  \end{center}

  \section{大学院生としての活動方針}
  \subsection{研究活動}
  今までの研究活動を振り返ってみたところ,自分の中で大きな熱量が発生していなかった\\

  なぜか?\\

  \underline{人から言われた通りの研究をしていたから}\\

  古池先生とmtgをしたりする中で,「自分なりにこうしたい!」と言えるところまでじっくり考えられていなかった部分がある.共著者に対して非常に失礼だと感じた.\\

  じゃあ,どうするべきか?\\

  \underline{自分の修士課程の研究についてのゴールを定める必要がある}\\

  もちろん,目標以上に進めることがいいことだが,一旦のゴールがないと何を目指しているのかがわからなくなる.自分はどこまでやり遂げたいのか,そのためにどのように動く必要があるのかなどを考えられるようにする.\\

  この点について詳細に考えるには時間が少し必要だと感じる.ALSTが終わってから考える.\\

  \clearpage

  \subsection{研究室での動きとメンター活動}
  メンター活動の目標は「勝井君を院生のメンティとして内山以上に成長させること」\\

  勝井君だけというよりは,直属のメンティとして勝井君とともに成長することを\underline{大目標}として,内山の強み?的なものを少しでも受け継げればいいなくらいに思ってる\underline{(小目標)}\\

  内山自身には人を成長させる力が\underline{少しは}あると思っているが,勝井君にも同じような能力を身につけてほしい(現に成長してきているが).\\

  メンティが「この人がメンターでよかった」と思えるような人になりたい.\\

  研究室での動きをもう少し「自主的」にしたい\\

  正直,学会の参加や原稿執筆などは想定外の話だと思っているので,その時は臨機応変に対応する.白髭先輩のように,自主的に問題を見つけて,何かしらの行動を起こすことは大切だと思う.現にバイト先では同じようなことができているが,研究室でできていないのは熱量の差だと思う.\\

  研究室での活動を「楽しいもの」にするのは大変だと思う.でも,「どのように自分を成長させるか」をベースに考えて,そのために「どのように動く必要があるか」を活動方針に組み込むことで,研究室で活動する目的が見えてくると思う(これを書いている時点ではALST関係で立て込んでいて,それが原因で嫌になってる説は高い).\\

  \clearpage

  \section{就職までの活動方針(就活)}
  研究活動が割と忙しい?気もしていて,自分が考える活動方針通りにいかないことがある.原因は結構明確で,\\

  \underline{研究とバイトと私生活の境目がなくなっていること}\\

  例えば,研究室を退出する時間もバラバラで,17:00に退出することもあれば,18:00前に退出することもある.早めにバイト先につけば,その分「今日は〇〇を目標にして動こう」と考える時間が生まれるが,それが最近はほとんどなくなっていた.\\

  17:00に研究室を退出する方向で考えて,しっかりとその日の動き方を考えて実践できるような動き方を目指す.そのためには,研究活動の時間を合わせるために,以下のどちらかを実施する必要がある.\\

  \underline{活動開始時間を早め,1日あたりの活動量を増やす}\\
  
  \underline{その日に行うタスクを細分化し,作業の効率を向上させる}\\

  個人的には後者が有力.前者だとだらけるし,寝る時間が削られる未来が見える.最近は特に夜の作業時間が増えていて,生活スケジュールにも影響が出ている.これは非常に大問題だと思っている.一旦後者ベースで考えて,10:00〜17:00で濃密に活動できるように慣れていく必要があると考えている.もちろん,メンティから資料の確認やmtgのお願いをされた場合は17:00以降も可能な限り対応しようと思うが,ベースは崩しすぎないように考える.\\

  \clearpage

  \section{両立・スケジュール管理の方針}
  10:00〜17:00 研究活動\\

  17:00〜22:20 バイト\\

  バイト先での話が楽しいのが原因で,帰るのが遅くなることがある.結局自分を苦しめているのを自覚して,スケジューリングをもっとしっかりやる.あと,土日は基本的に研究に触れない!これはしっかりと守れるようにする.学会などで緊急性が問われる時は除く.\\

  夜はしっかり寝ましょう.

  \clearpage
\end{document}
