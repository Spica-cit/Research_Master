\documentclass[a4paper,12pt]{ltjsarticle}
\usepackage{luatexja}
\usepackage{luatexja-fontspec}
\setmainjfont{IPAexMincho}
\usepackage{geometry}
\geometry{margin=25mm}

\usepackage{enumitem}
\usepackage{hyperref}
\usepackage{graphicx}
\usepackage{amsmath, amssymb}

\renewcommand{\baselinestretch}{1.2}
\setlength{\parskip}{0.5em}
\setlength{\parindent}{1em}

\title{メンターMTG}
\date{\today}

\begin{document}
  \maketitle

  \section*{内容}

  \textbf{ポスターについて} \par
  \begin{itemize}[label={--}]
    \item 構造的理解の話が欲しい
    \begin{itemize}
      \item 具体例を用いて話を説明したい
      \item 「220円から120円を引く」という考え方だけはダメ(間違っている訳ではない)
      \item 120円がどのように求められたのか?
      \item 鉛筆1本の値段はいくら? \par
            上記のように「段階的に」考える必要がある(考えた末に現れるのが構造的な考え方?)
      \item 構造的に考えるだけでは新たな問題が生じる. \par
            →知識を他の設問に転用できない \\
    \end{itemize} 
    \item 他の設問に転用するためにはどうするの? \par
          →抽象的に考える必要がある \\
    \item 研究目的 \par
          じゃあ,どうやって抽象化する?(これを考える) \par
          抽象化による問題構造の再利用を可能とする学習手法の提案 \par
          システム設計しました! \\
    \item 個人差が発生! \par
          →支援対象外の箇所を統一させる
  \end{itemize}
\end{document}
