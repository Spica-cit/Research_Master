\chapter{プログラミングの文章問題における問題解決過程の行き詰まり箇所の特定・解消システムの実装}

\section{まえがき}
本研究では,プログラミング課題における文章問題の問題解決過程をモデル化し,モデルに沿った学習支援システムの開発を行う.
開発しているシステムは,定式化構造作成画面,制約構造作成画面,解法構造作成画面,操作構造作成画面,処理構造作成画面の5つの画面で構成されている.
このうち,定式化構造作成画面は,これまで筆者ら(2)が開発したシステムの画面と同一の作業を学習者は行う.
また,制約構造作成画面と解法構造作成画面で学習者は,以前のシステムでは関係式で表現した処理部分に対して,抽象的な操作に変更した構造を作成する.
本システムでは,学習者が作成中の構造より前に作成した構造を閲覧できるようになっている.
学習者は自身が作成した構造を見ながら,構造を作成することが可能となっている.
本稿では,以前のシステムから変更した制約構造作成画面,解法構造作成画面に加え,新たに追加した操作構造作成画面,処理構造作成画面を説明する.

\newpage

\section{定式化構造作成画面}
前のシステムよりかなり変わっているため内容未定

\newpage


\section{抽象的操作構造作成画面}
図11に制約構造作成画面を示す.
制約構造作成画面は,これまで筆者らが開発したシステムで関係式を求めていた部分を,抽象的な操作を求める過程に変更している.
これにより学習者は,後述する操作構造作成画面で具体的な操作であるソースコードを求めることが可能になる.
この画面では,属性と抽象的な操作を示したひな型を学習者に提示される.
学習者はひな型に当てはまる属性を選択することが求められる.
これにより,解を求めるために必要な処理を表現する制約構造を作成する.

例えば図11の1では,「□と□のうち,値が大きい方を□に代入する」というひな型を提示している.
学習者はひな型に当てはまる「E(Aさんのテストの合計点)」,「F(Bさんのテストの合計点)」,「G(AさんとBさんの合計点の高い値)」を選択している.

制約構造作成画面では,入力属性と出力属性を正答と比較し,誤答した学習者に対してフィードバックを用いて支援する.
出力属性が誤っている場合は,システムが「出力属性が誤っているものがあります」と表示する.
また,入力属性が誤っている場合は,システムが「以下の操作の入力属性が誤っています」と表示し,入力属性が誤っている関係式を一覧で表示する.
例えば,図11の2では,学習者が「A(Aさんの国語の点数)」,「C(Bさんの国語の点数)」,「E(Aさんの合計点)」を使用し,「AとCを足してEを求める」と入力している.
また,「B(Aさんの数学の点数)」,「D(Bさんの数学の点数)」,「F(Bさんの合計点)」を使用し,「BとDを足してFを求める」と入力している.
しかし,入力属性が誤っているため,システムは「以下の操作の入力属性が誤っています AとBを足してEを求める CとDを足してFを求める」とフィードバックをされる.
学習者はこのフィードバックにより,抽象的な操作のひな型にあてはまる入出力属性と,解を求めるために必要な抽象的操作を思考することにつながる.

\newpage


\section{解法構造作成画面}
図12に解法構造作成画面を示す.
解法構造作成画面は,これまで筆者らが作成したシステムで関係式を入力していた部分を,抽象的操作を入力する部分に変更している.
これは,制約構造が関係式を求める過程から,抽象的な操作を求める過程に変更されているため,解法構造の処理部分として学習者が入力できる内容は抽象的な操作となる.
この画面では,解法構造のひな型が学習者に提示される.学習者は解法構造のひな型に属性と抽象的な操作を当てはめることが要求される.
例えば図12の1では,問題の要求として「点数の高い方の値を出力する」というものがあるため,学習者は「G」を構造の終わりに入力している.
また,この値を求めるには,「EとFのうち,大きい方の値をGに求める」の操作が必要なため,「G」につながる操作として「if E>F then G=E else G=F」を入力している.

解法構造作成画面では,「求める属性」,「属性を求めるための抽象的な操作」,「抽象的な操作に使用する属性」を正答と比較し,誤答した学習者に対してフィードバックを用いて支援する.
例えば,図12の2では学習者が「EとFのうち,値が大きい方をGに代入する」に使用する属性を「B(Aさんの数学の点数)」と「F(Bさんの合計点)」と解答している.
しかし,正答は「E(Aさんの合計点)」と「F(Bさんの合計点)」を選択することであり,抽象的な操作に使用する属性が誤っている.
この場合,システムは下から正誤判定を行い,正解の場合はコンボボックスの色を青色に変更する.
そして,誤っている箇所を発見されてから,「入力属性が間違っています」とフィードバックを表示し,誤っているコンボボックスの色を赤色に変更する.
学習者はこのフィードバックにより,最終的に求める属性からどの抽象的な操作や属性が誤りなのかと,抽象的な操作を使用した解を求めるための正しい一連の流れを思考することにつながる.

\newpage


\section{具体的操作構造作成画面}
図13に操作構造作成画面を示す.操作構造作成画面では,「if(?>?)」のようなソースコードのひな型と,学習者が制約構造で作成した「EとFのうち,大きい方の値をGに求める」のような抽象的な操作が学習者に提示する.学習者はひな型を組み合わせ,変数や値を入力することが要求される.

例えば図13の1では,「EとFのうち,値が大きい方をGに代入する」を構成するコードを作成してくださいと提示されている.学習者はこの抽象的操作を具体的操作で表現するために,「if(?>?)」を選択し,そして条件として,「E>F」を入力している.次にifの中の処理として「?=?」を選択し「G=E」を入力している.また,else文として「else」のブロックを生成しており,elseの中の処理として,「?=?」を選択し,「G=F」を入力している.「G=E」と「G=F」は条件分岐の入れ子となっているため,各ブロックの端にある矢印ボタンを押すことにより,ブロックが左右に移動し,入れ子の表現を行う.

操作構造作成画面では,正答と学習者の解答を比較し,誤答した学習者に対してフィードバックを用いて支援する.具体的には「使用するひな形」,「当てはまる属性」,「ブロックの位置」の3つを比較する.例えば,図13の2では,学習者がif文の条件に使用する属性を「B(Aさんの数学の点数)」と「D(Bさんの数学の点数)」と解答している.しかし,if文に使用する属性が誤っている.このように,使用する属性が誤っている場合,「選択している属性が誤っています」とフィードバックされ,誤っているコンボボックスの色を赤色に変更する.学習者はこのフィードバックにより,抽象的な操作を具体的な操作に変換するための正しい属性を思考することにつながる.


\newpage


\section{処理構造作成画面}
図14に処理構造作成画面を示す.この処理構造作成画面では,処理部分を暗黙的にした解法構造と,学習者が作成した具体的な操作が学習者に提示される.学習者は解法構造に具体的な操作を当てはめることが要求される.図14の1では,「E」と「F」から矢印が出ており,処理部分を介して「G」に繋がる解法構造を提示している.学習者は,「E」と「F」を使用し,「G」を求める処理として,「if E>F then G=E else G=F」のブロックを選択し,解法構造の中の青色で表示されている処理部分に当てはめている.

処理構造作成画面では具体的な操作について正答と学習者の解答を比較し,誤答した学習者に対してフィードバックを用いて支援する.例えば,図14の2のように学習者が「E(Aさんの合計点)」と「F(Bさんの合計点)」を入力属性として入力している.また,「G(AさんとBさんの合計点の高い方の値)」を出力属性とする際の具体的な操作として,「F=C+D」と選択している.しかし,正答は「if E>F then G=E else G=F」を選択することであり,具体的な操作が誤っている.この場合,システムは「ブロックの場所が誤っている箇所があります」とフィードバックを表示し,誤っている処理部分の色を赤色に変更する.このフィードバックにより,抽象的な操作を使用した解を求めるための正しい一連の流れを思考することにつながる.
