\documentclass[a4paper,12pt]{ltjsarticle}
\usepackage{luatexja}
\usepackage{graphicx}
\usepackage{amsmath, amssymb}
\usepackage{geometry}
\setmainjfont{IPAexMincho}
\setmainfont{Times New Roman}
\setsansjfont{IPAexMincho}
\setsansfont{Times New Roman}
\geometry{margin=25mm}

\begin{document}

\begin{titlepage}
  \centering
  {\large 千葉工業大学 東本研究室 \part}
  {\large 進捗報告書 \par}
  \vspace{1.5cm}
  {\Huge \bfseries 進捗報告 Ver.xx \par}
  \vspace{1cm}
  \rule{0.7\textwidth}{0.5pt} \par
  \vspace{1cm}
  {\Large 内山 裕太 \hspace*{1cm} \today \par}
  \vspace{3cm}
  \thispagestyle{empty}
\end{titlepage}

\section{今週実施したこと}

\section{実施事項の詳細}

\section{来週やること}

\section{現状の問題点}

\section{スケジュール}

\section{メンター関係}

\section{雑談}
$y = x$みたいな感じで、行中に埋め込めますし、以下のように書くこともできます。\\

\begin{equation}
  \int_{a}^{a}f\left(x\right)dx = 0
\end{equation}

複数行の数式も書けます。$=$の位置を揃えることもできます。\\

\begin{align}
  \int_{1}^{2}\left(x^2 + 3x\right)dx + \int_{1}^{2}\left(x^2 - 3x\right)dx &= \int_{1}^{2}\left\{\left(x^2 + 3x\right) + \left(x^2 - 3x\right)\right\}dx \\
  &= \int_{1}^{2}2x^2dx \\
  &= 2\left[\frac{x^3}{3}\right]^2_1 = \frac{2\left(2^3 - 1^3\right)}{3} = \frac{14}{3}
\end{align}

\end{document}